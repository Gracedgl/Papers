%%%%%%%%%%%%%%%%%%%%%%%%%%%%%%%%%%%%%%%%%%%%%%%%%%%%%%%%%%%%%%%%%%%%%%%%
%%%%%%%%%%%%%%%%%%%%%% Simple LaTeX CV Template %%%%%%%%%%%%%%%%%%%%%%%%
%%%%%%%%%%%%%%%%%%%%%%%%%%%%%%%%%%%%%%%%%%%%%%%%%%%%%%%%%%%%%%%%%%%%%%%%

%%%%%%%%%%%%%%%%%%%%%%%%%%%%%%%%%%%%%%%%%%%%%%%%%%%%%%%%%%%%%%%%%%%%%%%%
%% NOTE: If you find that it says                                     %%
%%                                                                    %%
%%                           1 of ??                                  %%
%%                                                                    %%
%% at the bottom of your first page, this means that the AUX file     %%
%% was not available when you ran LaTeX on this source. Simply RERUN  %%
%% LaTeX to get the ``??'' replaced with the number of the last page  %%
%% of the document. The AUX file will be generated on the first run   %%
%% of LaTeX and used on the second run to fill in all of the          %%
%% references.                                                        %%
%%%%%%%%%%%%%%%%%%%%%%%%%%%%%%%%%%%%%%%%%%%%%%%%%%%%%%%%%%%%%%%%%%%%%%%%

%%%%%%%%%%%%%%%%%%%%%%%%%%%% Document Setup %%%%%%%%%%%%%%%%%%%%%%%%%%%%

% Don't like 10pt? Try 11pt or 12pt
\documentclass[10pt]{article}

% The automated optical recognition software used to digitize resume
% information works best with fonts that do not have serifs. This
% command uses a sans serif font throughout. Uncomment both lines (or at
% least the second) to restore a Roman font (i.e., a font with serifs).
%\usepackage{times}
%\renewcommand{\familydefault}{\sfdefault}

% This is a helpful package that puts math inside length specifications
\usepackage{calc}

% Layout: Puts the section titles on left side of page
\reversemarginpar

%
%         PAPER SIZE, PAGE NUMBER, AND DOCUMENT LAYOUT NOTES:
%
% The next \usepackage line changes the layout for CV style section
% headings as marginal notes. It also sets up the paper size as either
% letter or A4. By default, letter was used. If A4 paper is desired,
% comment out the letterpaper lines and uncomment the a4paper lines.
%
% As you can see, the margin widths and section title widths can be
% easily adjusted.
%
% ALSO: Notice that the includefoot option can be commented OUT in order
% to put the PAGE NUMBER *IN* the bottom margin. This will make the
% effective text area larger.
%
% IF YOU WISH TO REMOVE THE ``of LASTPAGE'' next to each page number,
% see the note about the +LP and -LP lines below. Comment out the +LP
% and uncomment the -LP.
%
% IF YOU WISH TO REMOVE PAGE NUMBERS, be sure that the includefoot line
% is uncommented and ALSO uncomment the \pagestyle{empty} a few lines
% below.
%

%% Use these lines for letter-sized paper
\usepackage[paper=letterpaper,
            %includefoot, % Uncomment to put page number above margin
            marginparwidth=1.2in,     % Length of section titles
            marginparsep=.05in,       % Space between titles and text
            margin=1in,               % 1 inch margins
            includemp]{geometry}

%% Use these lines for A4-sized paper
%\usepackage[paper=a4paper,
%            %includefoot, % Uncomment to put page number above margin
%            marginparwidth=30.5mm,    % Length of section titles
%            marginparsep=1.5mm,       % Space between titles and text
%            margin=25mm,              % 25mm margins
%            includemp]{geometry}

%% More layout: Get rid of indenting throughout entire document
\setlength{\parindent}{0in}

\usepackage[shortlabels]{enumitem}

% Simpler bibsections for CV sections
% (thanks to natbib for inspiration)
%
% * For lists of references with hanging indents and no numbers:
%
%   \begin{bibsection}
%       \item ...
%   \end{bibsection}
%
% * For numbered lists of references (with hanging indents):
%
%   \begin{bibenum}
%       \item ...
%   \end{bibenum}
%
%   Note that bibenum numbers continuously throughout. To reset the
%   counter, use
%
%   \restartlist{bibenum}
%
%   at the place where you want the numbering to reset.

\makeatletter
\newlength{\bibhang}
\setlength{\bibhang}{1em}
\newlength{\bibsep}
 {\@listi \global\bibsep\itemsep \global\advance\bibsep by\parsep}
\newlist{bibsection}{itemize}{3}
\setlist[bibsection]{label=,leftmargin=\bibhang,%
        itemindent=-\bibhang,
        itemsep=\bibsep,parsep=\z@,partopsep=0pt,
        topsep=0pt}
\newlist{bibenum}{enumerate}{3}
\setlist[bibenum]{label=[\arabic*],resume,leftmargin={\bibhang+\widthof{[999]}},%
        itemindent=-\bibhang,
        itemsep=\bibsep,parsep=\z@,partopsep=0pt,
        topsep=0pt}
\let\oldendbibenum\endbibenum
\def\endbibenum{\oldendbibenum\vspace{-.6\baselineskip}}
\let\oldendbibsection\endbibsection
\def\endbibsection{\oldendbibsection\vspace{-.6\baselineskip}}
\makeatother

%% Reference the last page in the page number
%
% NOTE: comment the +LP line and uncomment the -LP line to have page
%       numbers without the ``of ##'' last page reference)
%
% NOTE: uncomment the \pagestyle{empty} line to get rid of all page
%       numbers (make sure includefoot is commented out above)
%
\usepackage{fancyhdr,lastpage}
\pagestyle{fancy}
%\pagestyle{empty}      % Uncomment this to get rid of page numbers
\fancyhf{}\renewcommand{\headrulewidth}{0pt}
\fancyfootoffset{\marginparsep+\marginparwidth}
\newlength{\footpageshift}
\setlength{\footpageshift}
          {0.5\textwidth+0.5\marginparsep+0.5\marginparwidth-2in}
\lfoot{\hspace{\footpageshift}%
       \parbox{4in}{\, \hfill %
                    \arabic{page} of \protect\pageref*{LastPage} % +LP
%                    \arabic{page}                               % -LP
                    \hfill \,}}

% Finally, give us PDF bookmarks
\usepackage{color,hyperref}
\definecolor{darkblue}{rgb}{0.0,0.0,0.3}
\hypersetup{colorlinks,breaklinks,
            linkcolor=darkblue,urlcolor=darkblue,
            anchorcolor=darkblue,citecolor=darkblue}

%%%%%%%%%%%%%%%%%%%%%%%% End Document Setup %%%%%%%%%%%%%%%%%%%%%%%%%%%%


%%%%%%%%%%%%%%%%%%%%%%%%%%% Helper Commands %%%%%%%%%%%%%%%%%%%%%%%%%%%%

%%% HEADING AT TOP OF CURRICULUM VITAE

% The title (name) with a horizontal rule under it
% (optional argument typesets an object right-justified across from name
%  as well)
%
% Usage: \makeheading{name}
%        OR
%        \makeheading[right_object]{name}
%
% Place at top of document. It should be the first thing.
% If ``right_object'' is provided in the square-braced optional
% argument, it will be right justified on the same line as ``name'' at
% the top of the CV. For example:
%
%       \makeheading[\emph{Curriculum vitae}]{Your Name}
%
% will put an emphasized ``Curriculum vitae'' at the top of the document
% as a title. Likewise, a picture could be included:
%
%   \makeheading[\includegraphics[height=1.5in]{my_picutre}]{Your Name}
%
% the picture will be flush right across from the name.
\newcommand{\makeheading}[2][]%
        {\hspace*{-\marginparsep minus \marginparwidth}%
         \begin{minipage}[t]{\textwidth+\marginparwidth+\marginparsep}%
             {\large \bfseries #2 \hfill #1}\\[-0.15\baselineskip]%
                 \rule{\columnwidth}{1pt}%
         \end{minipage}}

%%% SECTION HEADINGS

% The section headings. Flush left in small caps down pseudo-margin.
%
% Usage: \section{section name}
\renewcommand{\section}[1]{\pagebreak[3]%
    \vspace{1.3\baselineskip}%
    \phantomsection\addcontentsline{toc}{section}{#1}%
    \noindent\llap{\scshape\smash{\parbox[t]{\marginparwidth}{\hyphenpenalty=10000\raggedright #1}}}%
    \vspace{-\baselineskip}\par}

%%% LISTS

% This macro alters a list by removing some of the space that follows the list
% (is used by lists below)
\newcommand*\fixendlist[1]{%
    \expandafter\let\csname preFixEndListend#1\expandafter\endcsname\csname end#1\endcsname
    \expandafter\def\csname end#1\endcsname{\csname preFixEndListend#1\endcsname\vspace{-0.6\baselineskip}}}

% These macros help ensure that items in outer-type lists do not get
% separated from the next line by a page break
% (they are used by lists below)
\let\originalItem\item
\newcommand*\fixouterlist[1]{%
    \expandafter\let\csname preFixOuterList#1\expandafter\endcsname\csname #1\endcsname
    \expandafter\def\csname #1\endcsname{\csname preFixOuterList#1\endcsname\let\oldItem\item\def\item{\pagebreak[2]\oldItem}}
    \expandafter\let\csname preFixOuterListend#1\expandafter\endcsname\csname end#1\endcsname
    \expandafter\def\csname end#1\endcsname{\let\item\oldItem\csname preFixOuterListend#1\endcsname}}
\newcommand*\fixinnerlist[1]{%
    \expandafter\let\csname preFixInnerList#1\expandafter\endcsname\csname #1\endcsname
    \expandafter\def\csname #1\endcsname{\let\oldItem\item\let\item\originalItem\csname preFixInnerList#1\endcsname}
    \expandafter\let\csname preFixInnerListend#1\expandafter\endcsname\csname end#1\endcsname
    \expandafter\def\csname end#1\endcsname{\csname preFixInnerListend#1\endcsname\let\item\oldItem}}

% An itemize-style list with lots of space between items
%
% Usage:
%   \begin{outerlist}
%       \item ...    % (or \item[] for no bullet)
%   \end{outerlist}
\newlist{outerlist}{itemize}{3}
    \setlist[outerlist]{label=\enskip\textbullet,leftmargin=*}
    \fixendlist{outerlist}
    \fixouterlist{outerlist}

% An environment IDENTICAL to outerlist that has better pre-list spacing
% when used as the first thing in a \section
%
% Usage:
%   \begin{lonelist}
%       \item ...    % (or \item[] for no bullet)
%   \end{lonelist}
\newlist{lonelist}{itemize}{3}
    \setlist[lonelist]{label=\enskip\textbullet,leftmargin=*,partopsep=0pt,topsep=0pt}
    \fixendlist{lonelist}
    \fixouterlist{lonelist}

% An itemize-style list with little space between items
%
% Usage:
%   \begin{innerlist}
%       \item ...    % (or \item[] for no bullet)
%   \end{innerlist}
\newlist{innerlist}{itemize}{3}
    \setlist[innerlist]{label=\enskip\textbullet,leftmargin=*,parsep=0pt,itemsep=0pt,topsep=0pt,partopsep=0pt}
    \fixinnerlist{innerlist}

% An environment IDENTICAL to innerlist that has better pre-list spacing
% when used as the first thing in a \section
%
% Usage:
%   \begin{loneinnerlist}
%       \item ...    % (or \item[] for no bullet)
%   \end{loneinnerlist}
\newlist{loneinnerlist}{itemize}{3}
    \setlist[loneinnerlist]{label=\enskip\textbullet,leftmargin=*,parsep=0pt,itemsep=0pt,topsep=0pt,partopsep=0pt}
    \fixendlist{loneinnerlist}
    \fixinnerlist{loneinnerlist}

%%% EXTRA SPACE

% To add some paragraph space between lines.
% This also tells LaTeX to preferably break a page on one of these gaps
% if there is a needed pagebreak nearby.
\newcommand{\blankline}{\quad\pagebreak[3]}
\newcommand{\halfblankline}{\quad\vspace{-0.5\baselineskip}\pagebreak[3]}

%%% FORMATTING MACROS

% Uses hyperref to link DOI
\newcommand\doilink[1]{\href{http://dx.doi.org/#1}{#1}}
\newcommand\doi[1]{doi:\doilink{#1}}

% For \url{SOME_URL}, links SOME_URL to the url SOME_URL
\providecommand*\url[1]{\href{#1}{#1}}
% Same as above, but pretty-prints SOME_URL in teletype fixed-width font
\renewcommand*\url[1]{\href{#1}{\texttt{#1}}}

% For \email{ADDRESS}, links ADDRESS to the url mailto:ADDRESS
\providecommand*\email[1]{\href{mailto:#1}{#1}}
% Same as above, but pretty-prints ADDRESS in teletype fixed-width font
%\renewcommand*\email[1]{\href{mailto:#1}{\texttt{#1}}}

%\providecommand\BibTeX{{\rm B\kern-.05em{\sc i\kern-.025em b}\kern-.08em
%    T\kern-.1667em\lower.7ex\hbox{E}\kern-.125emX}}
%\providecommand\BibTeX{{\rm B\kern-.05em{\sc i\kern-.025em b}\kern-.08em
%    \TeX}}
\providecommand\BibTeX{{B\kern-.05em{\sc i\kern-.025em b}\kern-.08em
    \TeX}}
\providecommand\Matlab{\textsc{Matlab}}

% Custom hyphenation rules for words that LaTeX has trouble with
\hyphenation{bio-mim-ic-ry bio-in-spi-ra-tion re-us-a-ble pro-vid-er}

%%%%%%%%%%%%%%%%%%%%%%%% End Helper Commands %%%%%%%%%%%%%%%%%%%%%%%%%%%

%%%%%%%%%%%%%%%%%%%%%%%%% Begin CV Document %%%%%%%%%%%%%%%%%%%%%%%%%%%%

\begin{document}
\makeheading{William Curran}

\section{Contact Information}

% NOTE: Mind where the & separators and \\ breaks are in the following
%       table. Table is one row made up of three parboxes. The left
%       parbox has address info, the middle parbox has a vertical bar,
%       and the right parbox has phone and electronic contact
%       information.
%
% MACROS: \rcollength is the width of the right column of the table
%             (adjust it to your liking; default is 1.85in).
%         \spacewidth is width of area between left and right boxes.
%         \spacechar is character used to produce perforated vertical
%             boundary between boxes.
%
\newlength{\rcollength}\setlength{\rcollength}{2.85in}%
\newlength{\spacewidth}\setlength{\spacewidth}{20pt}
\newcommand\spacechar{$|$}
%
\begin{tabular}[t]{@{}p{\textwidth-\rcollength-\spacewidth}@{}p{\spacewidth}@{}p{\rcollength}}%

% Address box
\parbox{\textwidth-\rcollength-\spacewidth}{%
William Curran\\
2600 NW Century Dr. Apt 229\\
Corvallis, OR 97330
}

% Cheesy perforated vertical bar between boxes
% Shorten by removing \spacechar's
&  &

% Non-snail-mail contact information
\parbox{\rcollength}{%
\textit{Mobile:}             541-670-1382      \\
\textit{E-mail:} curranw@onid.orst.edu}
\end{tabular}

\textit{Website:} http://oregonstate.edu/$\sim$curranw \\
%%
%% In modern CV's, it seems like ``Objective'' is frowned upon. Instead,
%% incorporate it into a well-constructed cover letter. The ``More
%% information'' can go at the end of the CV, but it should not distract
%% from the section giving references available to contact.
%%
%
% \section{Objective}
%
% Placement in an academic position (i.e., faculty, postdoctoral, or
% research scientist) that allows for advanced research in distributed
% complex adaptive systems (i.e., modeling, analysis, design, and
% verification) with a particular focus on the control of engineered
% agents (e.g., for communications, control, software, electronics, and
% sustainability) and the analysis of biological phenomena (e.g.,
% self-organization, ecological rationality)
% \begin{innerlist}
% \item More information and auxiliary documents can be found at\\\url{http://www.tedpavlic.com/facjobsearch/}
% \end{innerlist}

\section{Research Interests}

Machine Learning, Multiagent Reinforcement Learning, Learning-Based Controls, Human-Robot Interaction.

%\textbf{Complex adaptive systems with applications in control systems
%engineering and behavioral science:} distributed algorithms, agent-based
%modeling, hybrid dynamic systems, amorphous computing, autonomous
%systems, control, communications, verification, cooperation,
%optimization, game theory, resource allocation, parallel computation,
%robotics, behavioral ecology, engineering education, bio-mimicry and
%bio-inspiration

\section{Education}

\textbf{Oregon State University},
Corvallis, OR
\begin{outerlist}

\item[] PhD,
        School of Mechanical, Industrial and Manufacturing Engineering, Expected 2016
        \begin{innerlist}
        \item Adviser:
              Professor Bill Smart
        \item Area of Study: Human-Robot Interaction and Learning from Demonstration
        \end{innerlist}
        
\item[] M.S.,
        School of Electrical Engineering and Computer Science, June 2012
        \begin{innerlist}
        \item Adviser:
              Professor Kagan Tumer
        \item Area of Study: Multi-Agent Systems and Learning Based Controls
        \end{innerlist}

\item[] B.S.,
        School of Electrical Engineering and Computer Science, June 2010
        \begin{innerlist}
        \item Emphasis on Machine Learning and Artificial Intelligence
        \item \emph{Cum Laude}
        \end{innerlist}

\end{outerlist}

% Add a little space to nudge next ``Conference Publications'' marginpar
% down to make room for tall ``Submitted Journal Publications''
% marginpar. If there are enough submitted journal publications, this
% space will not be needed (and should be removed).
\vspace{0.1in}

\section{Conference Publications}

\begin{bibenum}
	\item Curran, W., Brys, T., Taylor, M, and Smart, B.
    Using PCA to Efficiently Represent State Spaces.
    In: \emph{International Conference on Machine Learning 2015. European Workshop on Reinforcement Learning}  July 06--11, 2015 
    
	\item Curran, W., Thornton, T., Arvey, B, and Smart, B.
    Evaluating Impact in the ROS Ecosystem.
    In: \emph{IEEE International Conference on Robotics and Automation 2015} May 26--30, 2015
    
	\item Colby, M., Curran, W., Rebhuhn, C., and Tumer, K.
    Approximating Difference Evaluations with Local Knowledge.
    In: \emph{International Conference on Autonomous Agents and Multiagent Systems 2015} May 4--8, 2015
    
	\item Curran, W.
    Developing Learning from Demonstration Techniques for Individuals with Physical Disabilities.
    In: \emph{International Conference on Human-Robot Interaction. HRI Pioneers Workshop} March 2--5, 2015  

	\item Curran, W., Agogino, A., Tumer, K.
    Agent Partitioning with Reward/Utility-Based Impact
    In: \emph{AAAI Conference on Artificial Intelligence 2015. Multiagent Interaction without Prior Coordination Workshop} Jan 25--30, 2015
    
	\item Curran, W., Arvey, B., Thornton, T. and Smart, B.
    The ROS Ecosystem: Impacts, Insights and Improvements.
    In: \emph{ROSCon 2014} September 12--13, 2014  

	\item Lazewatsky, D., Bowie C., Curran, W., LaFortune, J., Narin, B., Nguyen, D., Wyman, A., Smart, W.
    Wearable Computing to Enable Robot Microinteractions.
    In: \emph{Robot and Human Interactive Communication 2014} August 25--29, 2014
   
    \item Curran, W., Agogino, A., Tumer, K.
    Hierarchical Simulation for Complex Domains: Air Traffic Flow Management.
    In: \emph{Genetic and Evolutionary Computation Conference 2014} July 12--16, 2014   

    \item Curran, W., Agogino, A., Tumer, K.  
    Using Reward/Utility Based Impact Scores in Partitioning.
    In: \emph{International Conference on Autonomous Agents and Multiagent Systems 2013} May 5--9, 2014   
    
    \item Curran, W., Agogino, A., Tumer, K.  
    Partitioning Agents and Shaping Their Evaluation Functions in Air Traffic Problems with Hard Constraints.
    In: \emph{Genetic and Evolutionary Computation Conference 2013} July 6--10, 2013
    
    \item Curran, W., Agogino, A., Tumer, K.
    Addressing Hard Constraints in the Air Traffic Problem through Partitioning and Difference Rewards.
    In: \emph{International Conference on Autonomous Agents and Multiagent Systems 2013} May 6--10, 2013
    
    \item Curran, W., Moore, T., Kulesza, T., Wong, W-K., Todorovic, S., Stumpf, S., White, R and Burnett, M.
        Towards recognizing ``cool'': can end users help computer vision recognize subjective attributes of objects in images?
        In: \emph{Proceedings of the 2012 ACM international conference on Intelligent User Interfaces},
        February 14--17, 2012.
        
    \item Shinsel, S., Kulesza, T., Burnett, M., Curran, W., Groce, A., Stumpf, S. and Wong, W-K.
        Mini-Crowdsourcing End-User Assessment of Intelligent Assistants: A Cost-Benefit Study.
        In: \emph{Proceedings of the 2011 Symposium on Visual Languages and Human-Centric Computing}, September 18--22, 2011. 
 
\end{bibenum}
%\section{Papers in Preparation}

%\begin{bibenum}

%\end{bibenum}
\halfblankline
\\
\section{Experience}

\textbf{Oregon State University},
Corvallis, OR
\begin{outerlist}
\item[] \textbf{Personal Robotics Lab}%
        \hfill \textit{September 2013 to Current}
	\begin{innerlist}
		\item Developing learning from demonstration algorithms for severely disabled individuals with amyotrophic lateral sclerosis and quadriplegia. This involves looking into effective interfaces for those with severe disabilities, developing shared autonomy learning algorithms, employing multi-robot coordination for efficient task execution, and how to learn from highly suboptimal human demonstrations. 
		\begin{innerlist}
			\item Dr. Bill Smart, Oregon State, MIME Department
		\end{innerlist}
	\end{innerlist}
		
\halfblankline


\item[] \textbf{Autonomous Agents and Distributed Intelligence}%
        \hfill \textit{Winter 2011 to June 2013}
	\begin{innerlist}
		\item Currently working on Air Traffic Control with a focus on multi agent reinforcement learning. This involves learning and using the air traffic simulator FEATS with the goal to apply reinforcement learning techniques in an effort to alleviate the air traffic congestion problem. Current work involves removing congestion and optimizing delay in the national airspace by controlling ground delay imposed upon aircraft.
		\begin{innerlist}
			\item Dr. Kagan Tumer, Oregon State, MIME Department
		\end{innerlist}
	\end{innerlist}
		
\halfblankline

\item[] \textbf{Roost Tracking Project}%
    \hfill \textit{Summer 2010 to Summer 2012}
    \begin{innerlist}
        \item Applied Machine Learning techniques on the image processing of multiple radar images. The goal was to design software to autonomously find rare bird roosts using radar images.
        \begin{innerlist}
            \item Dr. Dan Sheldon, Oregon State, EECS Department
            \item Dr. Tom Dietterich, Oregon State, EECS Department
        \end{innerlist}
    \end{innerlist}

\item[] \textbf{Vision Feedback }%
        \hfill \textit{Spring 2011 to Fall 2011}
        \begin{innerlist}
        	\item Lead a study exploring how users give feedback to machine learning vision algorithms. Lack of trust is a large issue in the Vision domain of Machine Learning. This study is designed to help develop techniques to alleviate this problem.
        	\begin{innerlist}
        		\item Dr. Weng-Keen Wong, Oregon State, EECS Department
        		\item Dr. Sinisa Todorovic, Oregon State, EECS Department
        	\end{innerlist}
        \end{innerlist}


\item[] \textbf{EUSES End User Development}
        \hfill \textit{Fall 2010 to Fall 2011}
\begin{innerlist}
    \item Explored the juncture between machine learning, human-computer interaction, and testing.
    \begin{innerlist}
    \item Dr. Margaret Burnett, Oregon State, EECS Department
    \item Dr. Alex Groce, Oregon State, EECS Department
    \end{innerlist}
\end{innerlist}
\end{outerlist}


\textbf{Navy Research Lab},
Washington, DC
\begin{outerlist}

\item[] \textbf{NRL Student Contractor}%
        \hfill \textit{June 2015 to September 2015}
	\begin{innerlist}
		\item Developed the Dimensionality Reduced Reinforcement Learning (DRRL) algorithm for learning in high-dimensional spaces. It uses dimensionality reduction on demonstrations to compute a low-dimensional manifold in which to learn in.
		\begin{innerlist}
			\item Dr. David Aha, Navy Research Lab, Navy Center for Applied Research in Artificial Intelligence
		\end{innerlist}
	\end{innerlist}
\end{outerlist}

\textbf{NASA AMES}
Moffet Field, Mountainview, CA
\begin{outerlist}

\item[] \textbf{NASA AMES Intern}%
        \hfill \textit{June 2013 to September 2013}
	\begin{innerlist}
		\item Developed a new simulation approach leveraging the concept of a multi-fidelity simulation. Computationally complex simulation is used only when agents are in a potential separation violation, while a low fidelity simulation is used for navigation. The key research focus is when to change from one fidelity to another, as small nuanced issues can easily occur that impedes agent learning.
		\begin{innerlist}
			\item Dr. Adrian Agogino, NASA AMES, Moffett Field
		\end{innerlist}
	\end{innerlist}


\item[] \textbf{NASA AMES Intern}%
        \hfill \textit{June 2012 to September 2012}
	\begin{innerlist}
		\item Extended my work on Air Traffic Control with adviser Adrian Agogino. Here I extended previous Air Traffic Control work by simulating the entire NAS (35,000 planes) rather than a small subset of the NAS, and developed an automated partitioning system for aircraft to learn quickly and efficiently.
		\begin{innerlist}
			\item Dr. Adrian Agogino, NASA AMES, Moffett Field
		\end{innerlist}
	\end{innerlist}
		
\halfblankline
\end{outerlist}

\section{Invited Talks}
\restartlist{bibenum}
\begin{bibenum}
	\item Curran, W.
    Overview of HRI at Oregon State University
    \emph{Washington State University} Mar 26, 2015
    
	\item Curran, W., Thornton, T., Arvey, B, and Smart, B.
    Evaluating Impact in the ROS Ecosystem.
    \emph{ROSCon} Sep 12--13, 2014

\end{bibenum}
\section{Expertise}

%\textit{Software Experience}:

%\begin{innerlist}
%	\item FEATS: Fast Event-based Air Traffic Simulator
%	\\
%\end{innerlist}

\textit{Computer Science}:
\begin{innerlist}
\setlength{\itemindent}{3em}
	\item Multi-agent Reinforcement Learning
	\item Machine Learning
	\item Software Engineering
	\item Human-Computer Interaction
	\item Human-Robot Interaction
	\\
\end{innerlist}

\end{document}

%%%%%%%%%%%%%%%%%%%%%%%%%% End CV Document %%%%%%%%%%%%%%%%%%%%%%%%%%%%%

%----------------------------------------------------------------------%
% The following is copyright and licensing information for
% redistribution of this LaTeX source code; it also includes a liability
% statement. If this source code is not being redistributed to others,
% it may be omitted. It has no effect on the function of the above code.
%----------------------------------------------------------------------%
% Copyright (c) 2007, 2008, 2009, 2010, 2011 by Theodore P. Pavlic
%
% Unless otherwise expressly stated, this work is licensed under the
% Creative Commons Attribution-Noncommercial 3.0 United States License. To
% view a copy of this license, visit
% http://creativecommons.org/licenses/by-nc/3.0/us/ or send a letter to
% Creative Commons, 171 Second Street, Suite 300, San Francisco,
% California, 94105, USA.
%
% THE SOFTWARE IS PROVIDED "AS IS", WITHOUT WARRANTY OF ANY KIND, EXPRESS
% OR IMPLIED, INCLUDING BUT NOT LIMITED TO THE WARRANTIES OF
% MERCHANTABILITY, FITNESS FOR A PARTICULAR PURPOSE AND NONINFRINGEMENT.
% IN NO EVENT SHALL THE AUTHORS OR COPYRIGHT HOLDERS BE LIABLE FOR ANY
% CLAIM, DAMAGES OR OTHER LIABILITY, WHETHER IN AN ACTION OF CONTRACT,
% TORT OR OTHERWISE, ARISING FROM, OUT OF OR IN CONNECTION WITH THE
% SOFTWARE OR THE USE OR OTHER DEALINGS IN THE SOFTWARE.
%----------------------------------------------------------------------%